\documentclass{letter}
%\documentclass{article}

\usepackage[utf8]{inputenc}
\usepackage{hyperref,cite}
\usepackage[a4paper, total={7in, 10in}]{geometry}
%\signature{Daniel Antunes, Jacques-Olivier Lachaud, Hugues Talbot}
%\address{}

\makeatletter
\newenvironment{thebibliography}[1]
     {\list{\@biblabel{\@arabic\c@enumiv}}%
           {\settowidth\labelwidth{\@biblabel{#1}}%
            \leftmargin\labelwidth
            \advance\leftmargin\labelsep
            \usecounter{enumiv}%
            \let\p@enumiv\@empty
            \renewcommand\theenumiv{\@arabic\c@enumiv}}%
      \sloppy
      \clubpenalty4000
      \@clubpenalty \clubpenalty
      \widowpenalty4000%
      \sfcode`\.\@m}
     {\def\@noitemerr
       {\@latex@warning{Empty `thebibliography' environment}}%
      \endlist}
\newcommand\newblock{\hskip .11em\@plus.33em\@minus.07em}
\makeatother

\signature{Daniel Martins Antunes, Jacques-Olivier Lachaud, Hugues Talbot}
\begin{document}
\begin{letter}{Cover letter for paper ``Elastica Energy Regularization via Graph Cuts''}
\opening{Dear editors,}

Our article is an extension of DGMM'21 paper ``A Maximum-Flow Model
for Digital Elastica Shape Optimization''~\cite{antunes21}. We have
included two new significant contributions. In section~5, we formally
establish the link between the curve-shortening flow and the balance
coefficient flow. The latter is described in the paper and it is of
fundamental importance in our segmentation algorithm.  In section~7 we
present the results of applying our segmentation algorithm on more
than 200 images of the Coco dataset~\cite{lin2014microsoft}. We reach
similar values of precision and recall metrics compared to the grabcut
algorithm~\cite{rother04grabcut}, but with a much smoother
contour. The elastica energy value is reduced by $90\%$ in average.
This is also reflected in the quality of achieved segmentations. Our
algorithm suffers much less from oversegmentation and is capable to
correctly complete the contours of objects that were considered
disjoint by grabcut. We also made available the source code of our
algorithm and a report showing the results for all the images in the
experiment.

Thank you for your time and consideration.

\closing{Sincerely yours,}

\bibliographystyle{plain}
\bibliography{../gf-paper.bib}

\end{letter}


%\end{letter}


\end{document}
